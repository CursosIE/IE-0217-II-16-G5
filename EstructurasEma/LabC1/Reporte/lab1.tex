\documentclass[11pt]{article}

\usepackage[letterpaper, margin=1in]{geometry}
\usepackage{listings}
\usepackage[spanish]{babel}
\usepackage[utf8]{inputenc}
\usepackage{multirow}
\usepackage{tabularx}
\usepackage{longtable}



%Figuras
\usepackage{graphicx, subfigure}
\usepackage[]{tikz}
\usepackage{pbox}

%Matemática
\usepackage{amsmath}
\usepackage{amssymb}

%Símbolos mate extra (alfabetos, etc.)
\usepackage{mathrsfs}


%Algoritmos
\usepackage{float}
\usepackage{algorithm}
\usepackage{algorithmicx}
\usepackage{algpseudocode}
\usepackage{listings}


\usepackage{color}
\usepackage{hyperref}

\usepackage{mdframed}
\usepackage{tcolorbox}
\usepackage{multicol}
\usepackage{booktabs}
\usepackage{tabulary}
\definecolor{darkblue}{rgb}{0 , 0.054 , 0.196}



\title{Reporte de Laboratorio 0}
\author{Emmanuel - B51296}

\begin{document}

\maketitle
\hrule
\hrule
\tableofcontents
\hspace{5mm}
\hrule
\hrule


\section{Introducción}
Este laboratorio fue realizado con el fin de aprender y obtener habilidades para el manejo de punteros en C++, para de esta forma emplearlos de forma regular puesto a que son herramientas muy útiles en la programación.
\subsection{Objetivos}
\begin{itemize}
	\item Aprender acerca del uso de punteros en C++.
\end{itemize}

\section{Código}

\subsection{C++}
\begin{lstlisting}
#include <iostream>
#include "string"
#include <cstdio>
#include <cstdlib>
using namespace std;

int* codonpos(char** chain){
int* codonpos = new int[2];
codonpos = (int *)malloc(2 * sizeof(int));
int trig=0;
int i=0;
while(trig!=2){
if(*(*chain+i) == 'U'){
if(*(*chain +1+i) =='A'){
if(*(*chain +2+i) =='A'){
*(codonpos+trig)=i;
trig++;
}
if(*(*chain +2+i) =='G'){
*(codonpos+trig)=i;
trig++;
}
}
if(*(*chain +1+i)=='G'){
if(*(*chain +2+i)=='A'){
*(codonpos+trig)=i;
trig++;
}
}
}
i=i+3;
}
return codonpos;
}

char* traduce(char** chain){
int* arr = codonpos(chain);
int a=arr[0];
int b=arr[1];
int c= b-a+3;
char* aa = new char[c];
aa = (char *)malloc(c * sizeof(char));
int i = a;
int n = 0;
for(i=a; i<=b; i+=3){
if(*(*chain+i) == 'U'){
if(*(*chain +1+i) =='A'){
if(*(*chain +2+i) =='U'|| *(*chain +2+i) =='C'){
aa[n]='Y';
n++;
}
if(*(*chain +2+i) =='A' || *(*chain +2+i) =='G'){
aa[n]='-';
n++;
}
}
if(*(*chain +1+i) =='G'){
if(*(*chain +2+i) =='G'){
aa[n]='W';
n++;
}
if(*(*chain +2+i) =='A'){
aa[n]='-';
n++;
}
if(*(*chain +2+i) =='U'|| *(*chain +2+i) =='C'){
aa[n]='C';
n++;
}
}
if(*(*chain +1+i) =='C'){
if(*(*chain +2+i) =='U'|| *(*chain +2+i) =='C' || *(*chain +2+i) =='A' || *(*chain +2+i) =='G'){
aa[n]='S';
n++;
}
}
if(*(*chain +1+i) =='U'){
if(*(*chain +2+i) =='G'|| *(*chain +2+i) =='A'){
aa[n]='L';
n++;
}
if(*(*chain +2+i) =='U'|| *(*chain +2+i) =='C'){
aa[n]='F';
n++;
}
}
}
if(*(*chain+i) == 'G'){
if(*(*chain +1+i) =='G'){
if(*(*chain +2+i) =='U'|| *(*chain +2+i) =='C' || *(*chain +2+i) =='A' || *(*chain +2+i) =='G'){
aa[n]='G';
n++;
}
}
if(*(*chain +1+i) =='A'){
if(*(*chain +2+i) =='G'|| *(*chain +2+i) =='A'){
aa[n]='E';
n++;
}
if(*(*chain +2+i) =='U'|| *(*chain +2+i) =='C'){
aa[n]='D';
n++;
}
}
if(*(*chain +1+i) =='C'){
if(*(*chain +2+i) =='U'|| *(*chain +2+i) =='C' || *(*chain +2+i) =='A' || *(*chain +2+i) =='G'){
aa[n]='A';
n++;
}
}
if(*(*chain +1+i) =='U'){
if(*(*chain +2+i) =='U'|| *(*chain +2+i) =='C' || *(*chain +2+i) =='A' || *(*chain +2+i) =='G'){
aa[n]='V';
n++;
}
}
}
if(*(*chain+i) == 'A'){
if(*(*chain +1+i) =='G'){
if(*(*chain +2+i) =='G'|| *(*chain +2+i) =='A'){
aa[n]='R';
n++;
}
if(*(*chain +2+i) =='C'|| *(*chain +2+i) =='U'){
aa[n]='S';
n++;
}
}
if(*(*chain +1+i) =='A'){
if(*(*chain +2+i) =='G'|| *(*chain +2+i) =='A'){
aa[n]='K';
n++;
}
if(*(*chain +2+i) =='C'|| *(*chain +2+i) =='U'){
aa[n]='N';
n++;
}
}
if(*(*chain +1+i) =='C'){
if(*(*chain +2+i) =='U'|| *(*chain +2+i) =='C' || *(*chain +2+i) =='A' || *(*chain +2+i) =='G'){
aa[n]='T';
n++;
}
}
if(*(*chain +1+i) =='U'){
if(*(*chain +2+i) =='U'|| *(*chain +2+i) =='C' || *(*chain +2+i) =='A'){
aa[n]='I';
n++;
}
if(*(*chain +2+i) =='G'){
aa[n]='M';
n++;
}	
}

}
if(*(*chain+i) == 'C'){
if(*(*chain +1+i) =='G'){
if(*(*chain +2+i) =='U'|| *(*chain +2+i) =='C' || *(*chain +2+i) =='A' || *(*chain +2+i) =='G'){
aa[n]='R';
n++;
}	
}
if(*(*chain +1+i) =='A'){
if(*(*chain +2+i) =='G'|| *(*chain +2+i) =='A'){
aa[n]='Q';
n++;
}
if(*(*chain +2+i) =='C'|| *(*chain +2+i) =='U'){
aa[n]='H';
n++;
}
}
if(*(*chain +1+i) =='C'){
if(*(*chain +2+i) =='U'|| *(*chain +2+i) =='C' || *(*chain +2+i) =='A' || *(*chain +2+i) =='G'){
aa[n]='P';
n++;
}	
}
if(*(*chain +1+i) =='U'){
if(*(*chain +2+i) =='U'|| *(*chain +2+i) =='C' || *(*chain +2+i) =='A' || *(*chain +2+i) =='G'){
aa[n]='L';
n++;
}	
}	
}

}
return aa;

}

void imprimirVc(char* vector, int size){	
int i;
for (i = 0; i < size; i++) {
printf("%c", *(vector+i));		
}
printf("\n");
}

int main(int argc, char** argv){
argv++;
int* arreglo = codonpos(argv);
char* prueba = traduce(argv);
imprimirVc(prueba, 2); 
imprimirVc(prueba, 7);
delete prueba;

}
\end{lstlisting}


\section{Conclusiones}
Durante este laboratorio se adquirieron habilidades de programación realmente útiles, pues se aprendió sobre el uso de una de las herramientas más importantes y útiles que tiene un programador a su alcance, los punteros. Haciendo uso de estos, se pueden simplificar muchas tareas durante la realización de un programa debido a como simplifican el manejo de la memoria dinámica que estos ofrecen.




\end{document}
